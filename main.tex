\documentclass[12pt,a4paper]{article}
\usepackage[utf8]{vietnam}
\usepackage{enumerate}
\usepackage[margin=1in]{geometry}
% \usepackage{fixltx2e} %Provide \textsubscript
\usepackage{amsmath, amssymb, amsfonts}
% \usepackage{mathptmx} %Use Times New Roman fonts but not works perfectly
%\usepackage{pgfplots}
\usepackage{indentfirst}

\begin{document}
\begin{center}
    \textbf{Đề tài}
\end{center}

Tham khảo Jame Stewart, Troy Day, Biocalcuclus-Calculus for life sience.
    \begin{enumerate} [1.]
            \item Tìm hiểu về \textbf{Cerebral blood flow} trong ví dụ 4, phần 6.1.
            \item Tiềm hiểu về \textbf{Survival an Renewal; Cardiac output; Blood flow} phần 6.3.
    \end{enumerate}
\textbf{Yêu cầu:} Hiểu được các thuật ngữ và cách tính. Nêu ví dụ cho mỗi phần, không dùng lại ví dụ của sách.
\newpage
\begin{center}
    \textbf{Lời nói đầu}
\end{center}

    Giải tích 1 là một môn đại cương quan trọng đối với các sinh viên ngành Kỹ thuật – Công nghẹ nói chung và sinh viên trường Đại học Bách Khoa nói riêng. Vì thế, việc dành ra một khoảng thời gian lớn để học hỏi lý thuyết và nghiên cứu môn học này là một điều tất yếu, để các sinh viên có một nền móng vũng chắc trong các môn KHTN, tạo tiền đề cho sinh viên học tốt các môn chuyên ngành sau này.\newline
Ở Bài tập lớn này nhóm chúng em đã hiểu được các thuật ngữ và cách tính của:\\
+ Lưu lượng máu não\\
+ Sự tồn tại và thay mới\\
+ Cung lượng tim ( Lượng máu tim bơm ra trong một đơn vị thời gian)\\
+ Lưu lượng máu \\
\textbf{Nhận xét của GVHD:}

\newpage
\begin{center}
    \textbf{Lời cám ơn}
\end{center}

%\tableofcontents
\end{document}
